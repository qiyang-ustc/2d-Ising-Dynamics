% !TEX program = xelatex

\documentclass[a4paper]{article}
%% Language and font encodings
\usepackage[english]{babel}
\usepackage[utf8x]{inputenc}
\usepackage[T1]{fontenc}
\usepackage{wrapfig}
\usepackage{subcaption}
\usepackage{graphics}
\usepackage{booktabs}
\usepackage{multirow}
\usepackage[table]{xcolor}
\usepackage{amsmath}
\usepackage{amsthm}
\usepackage{amsfonts}
\usepackage{tweaklist}
\usepackage{blindtext}
%% Useful packages
\usepackage{amsmath}
\usepackage{graphicx}
\usepackage[colorinlistoftodos]{todonotes}
\usepackage[colorlinks=true, allcolors=blue]{hyperref}
\usepackage{xeCJK}
\usepackage{url}
%\usepackage{emumerates}
%% Sets page size and margins
\usepackage[a4paper,top=2cm,bottom=2cm,left=1cm,right=3cm,marginparwidth=2cm]{geometry}

\begin{document}
\setlength{\leftskip}{20pt}

\title{2D-Ising critical fluctuation}
\author{仰旗,学号:{\it 201928000807088}}
%%%%% ------------------------ %%%%% 

 \maketitle

% \begin{abstract}
% \end{abstract}
% \tableofcontents
\section{Introduction}
This project mainly investigate 2D-Ising system near the critical point. We use Markov Chain Monte Carlo method
with Metropolis updates as dynamics of this system and extrapolate the result from finite size system to infinity size
system by tools named Finite Size Scaling.
\newline
The 2D-Ising system is always a classical example in statistical physics. The most important property of this system 
near critical point is the space-correlation would be divergent and fluctuation with different span could be treatedd 
equivalently. In other words, all configurations are near equilibrium. As a result, we can do linear response analysis 
without any approximation. It is quite straightforward that the relaxtion behavior coule be related to information of 
fluctuation in critical points.
\newline
This project is coded with Julia and Python. The code is published on Github: 
\url{https://github.com/qiyang-ustc/2d-Ising-Dynamics}

\section{Method}
The model and the method in this report mainly follow the article given by Ito in 1993.\url{https://arxiv.org/abs/cond-mat/9302009}
In that article, Ito studied 3D-Ising model but did not related the result with fluctuation-dissipation theorem.

\section{Theoretical analysis}
The Hamiltonian of oußr system is: $s_i$ could be $+1$ or $-1$.
$$\mathcal{H}=\sum_{\left\langle i,j\right\rangle}-Js_is_j-\sum_ihs_i$$
We first prepare our system at $h\rightarrow\infty$, which is equivalent to say:
$$\mathcal{H} = \mathcal{H}_0-Nhm\theta(-t)$$
where N is the number of total spins and m is order parameter $m=\sum s_i/N$. T is time. $\theta(t)=1$ if $t>0$ else $theta(t)=0$
It is obvious that the expectation for order parameter is zero for non-disturbed state.
And the system is prepare at $m(0)=1$.\newline
Now, we use linear response theory to analyze this system: when $t=0$, The wight is 
$$W(x,t=0) = \frac{e^{-\beta H}}{Z}=\frac{e^{-\beta H_0}\times e^{\beta Nmh}}{\sum_{s}e^{-\beta H_0+\beta Nmh}}$$
So,
$$=\frac{e^{-\beta H_0}(1+\beta Nhm)}{Z}-\frac{e^{-\beta H_0}}{Z^2}e^{-\beta H_0}Nhm=W_0(x,0)\times(1+\beta Nh(m-\left\langle m\right\rangle))$$
We can conclude that:
$$\rho(t=0)+\Delta\rho(t=0)=W(x,0)=W_0(x,0)(1+\beta Nh\delta m) $$
$$\Delta\rho(t) = \beta NH\delta mW_0(t)$$
If we donated the time-correlation of order parameter by $A(t)$ which means:
$$A(t)=\left\langle\delta m(t)\delta m(0)\right\rangle$$
Generalized susceptibility $\chi$ in linear response theory:
$$\left\langle \delta m(t)\right\rangle=\int_{\infty}^{t}\chi(t-t')\theta(-t')Nhdt'$$
Besides, by definition
$$\left\langle \delta m(t)\right\rangle=Tr(\Delta\rho(t)\hat{m})=Tr(\Delta\rho(0)\hat{m}(t))=\beta Nh \left\langle m(t)m(0)\right\rangle=\beta NhA(t)$$
Take derivatives of the equation:
$$\beta NhA(t)=\left\langle m(t)\right\rangle=\int_{-\infty}^{t}\chi(t-t')Nh\theta(-t')dt'$$
We can get:
$$\chi(t)= -\beta\dot{A}(t)\quad if\quad t>0$$

\subsection{Dynamics}
The dynamics in this project is given by Heat Bath Method. Each steps, we pick up a spin sequentially or randomly.
The reason of using Heat Bath updating instead of Metropolis is a little subtle:
The differece between Heat Bath and Metropolis is, in Metropolis, the probability of updating is:
$$P_M(s_i\rightarrow -s_i)=min\{1,e^{-2\sum_{j \in \left\langle i,j \right\rangle}\beta Js_js_i}\}$$
While the probability of updating in Heat Bath is:
$$P_{HB}(s_i\rightarrow -s_i)=\frac{e^{\sum_{j \in \left\langle i,j \right\rangle}\beta Js_js_i}}{e^{\sum_{j \in \left\langle i,j \right\rangle}\beta Js_js_i}+e^{-\beta J\sum_{j \in \left\langle i,j \right\rangle}s_js_i}}$$
Though these two methods both satisfy Detailed Balance, comparing to Heat Bath, Metropolis expand the probability of 
updating by a const: $P_M(s_i\rightarrow -s_i))=P_{HB}(s_i\rightarrow -s_i)/P_{HB}(-s_i\rightarrow s_i)$. 
And $P_{HB}(-s_i\rightarrow s_i)$ is configuration depended. Metropolis actually accelerate the updating, what it does 
is equivalent to rescale time-axis.



% -----------------------------------Appendix----------------------------------------
%\newpage
% -----------------------------------REFERENCE----------------------------------------
\bibliographystyle{alpha}
% \bibliography{sample}
\end{document}